\section{Conclusão}
\label{sec:conclusao}


Neste trabalho foi constatado que é possível aplicar uma solução baseada em métodos ágeis e no pensamento lean sobre a gestão de um contrato público de desenvolvimento de software.

Possivelmente uma solução automatizada de análise estática de código, por exemplo, como a utilizada neste estudo de caso, e a exigência de entrega de uma suíte automatizada de testes poderia facilitar a vistoria técnica sobre o
 código fonte, onde a informação resultante dessa análise pudesse auxiliar a tomada de
 decisão sobre o faturamento das ordens de serviço. Isto corroboraria com o Art. 25, inciso III, alínea b da Instrução Normativa no 04 (BRASIL, 2010) que diz para realizar a
 "avaliação da qualidade dos serviços realizados ou dos bens entregues e justificativas, de
 acordo com os Critérios de Aceitação definidos em contrato, a cargo dos Fiscais Técnico
 e Requisitante do Contrato".

Assim, a solução desenvolvida pelo IPHAN, aplicada no projeto SICG, resultou na
 entrega do software requisitado ao final do contratado. Embora esta solução seja passível
 de melhorias, pois há indícios de que alguns resultados apresentados não tenham sido
 totalmente satisfatórios, podemos concluir que o problema de pesquisa deste estudo "Alguns contratos de desenvolvimento de software da organização não resultaram na entrega
 do software requisitado ao final do contrato", que ocorria frequentemente na autarquia, não
 ocorreu na gestão de contratos ágil aplicada.

Ainda, a solução de gestão de contratos definida pelo
 IPHAN não fere o que é determinado na Lei no 8.666/93 ou na IN 04/2010 ou nos Princípios da APF. Ao mesmo tempo, a solução foi baseada nos valores, princípios e práticas
 do pensamento lean e de metodologias ágeis.
 
Por fim, destacamos que essa aproximação da academia e uma organização pública, que contrata serviços de desenvolvimento de software, mostrou-se benéfica para ambas partes, e consequentemente para o estado.