\section[Metodologia de Pesquisa]{Metodologia de Pesquisa}

Na metodologia de pesquisa adotada neste trabalho, foram definidos: a natureza da pesquisa; o tipo de metodologia de pesquisa; o tipo de abordagem de pesquisa; os métodos de
procedimentos de pesquisa e os tipos de técnicas de coletas de dados.

O procedimento de pesquisa escolhido foi o estudo de caso. As técnicas de coleta de dados selecionadas foram
documentos, questionários e entrevistas informais. 

Um estudo de caso pode ser composto por 6 etapas: Plano, Projeto, Preparação, Coleta, Análise e Compartilhamento. Nesta pesquisa, o esquema adotado compreende as fases: Planejamento; Coleta; Análise, e Compartilhamento (\cite{yin}). Portanto, as estapas Plano, Projeto e Preparação definidas por Yin foram agrupadas na fase Planejamento nesse estudo de caso.

O Planejamento consiste na determinação da questão de pesquisa, a escolha da metodologia de pesquisa, a definição das fases da pesquisa,  a definição dos procedimentos de pesquisa e das técnicas de coleta de dados, a construção do referencial teórico e a proposta do trabalho final.


Na Coleta são executados os procedimentos de pesquisa e as técnicas de coletas de dados a seguir:

\begin{itemize}
\item Revisão Bibliográfica do aporte teórico sobre \textit{lean} e métodos ágeis a partir de livros, dissertações e trabalhos relacionados à área de pesquisa;
\item Estudo de Caso: utilizar um estudo de caso real de uma organização pública brasileira;
\item Entrevistas: dados serão coletados por meio de estrevistas informais, além de questionário, para incremento do estudo de caso;
\item Documentos: coleta de dados dos documentos dos processos fornecidos pela autarquia pública do estudo de caso será realizada para coleta de dados para análise.
\end{itemize}

A Análise diz respeito a fase em que os dados coletados serão analisados e interpretados. A análise compreende tanto a análise quantitativa quanto a análise qualiitativa.

O Compartilhamento diz respeito a redação dos resultados de forma adequado para o leitor alvo.