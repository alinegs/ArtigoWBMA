\begin{resumo}
Atualmente, algumas organizações da Administração Pública Federal iniciam investimentos para adotar contratações de serviços de desenvolvimento de software utilizando métodos ágeis, motivado pelo entendimento de que os instrumentos contratuais, hoje em vigor, apresentam limitações que causam impacto nos custos dos projetos e na entrega do produto. O objetivo desse trabalho foi analisar a influência da adoção de métodos ágeis na gestão de contratos públicos de desenvolvimento de software. Para isso, foi realizada uma investigação empírica, descritiva, por meio da execução de um estudo de caso exploratório. Analisamos os efeitos sobre a entrega de ordens de serviço, a qualidade interna do produto e a satisfação do cliente da solução desenvolvida na organização. Após a análise dos dados coletados concluímos que neste caso foi possível a aplicação de métodos ágeis na gestão de contratos públicos de desenvolvimento de software, e que isso pode melhorar a atividade da gestão do desenvolvimento, em que pese, haja restrições face ao normativo vigente.
{
\\
\\
\textbf{Palavras-chave:} Gerenciamento, contratações, \textit{software}, organizações públicas, \textit{lean}, métodos ágeis, desenvolvimento, produção, estudo de caso, engenharia de software empírica.}
\end{resumo}
