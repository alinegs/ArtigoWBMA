\section{Introdução}
\label{intro}

Nos últimos anos, tem surgido iniciativas isoladas de algumas entidades da Administração Pública, em adotar métodos ágeis em suas equipes de desenvolvimento, com destaque para o Scrum e o \textit{Extreme Programming}-XP \cite{TCU:2013} \cite{RTMAC}.  Ainda mais recente, houve a iniciativa de mesclar o uso de métodos ágeis com a filosofia de gestão da produção, conhecida como \textit{Lean}, com o foco em gerenciar o contrato dos fornecedores de desenvolvimento de \textit{software}. 


Essas teorias representam um contraponto a metodologias mais prescritivas e preditivas, também conhecidas como tradicionais, que possuem seu amparo teórico principalmente na visão da teoria da administração científica, que tem como um dos percursores Frederick W. Taylor e pressupõe que todo saber e tomadas de decisões são funções específicas da gerência e que os trabalhadores devem executar suas tarefas por meio de métodos e procedimentos pré-definidos \cite{administracao}.

O \textit{scrum} é uma metodologia ágil desenvolvida para a gestão do processo de desenvolvimento de \textit{software}. É uma abordagem que aplica idéias de controle do fluxo de trabalho, oriundo da indústria de manufatura, ao desenvolvimento de \textit{software}, resultando assim, numa abordagem que reintroduz a ideia de flexibilidade, adaptabilidade e produtividade. O scrum surgiu a partir do “Manifesto Ágil”, publicado em 2001, e como método ágil tem como valores: indivíduos e interações mais do que processos e ferramentas; \textit{software} funcionando para o cliente em vez de documentação; colaboração com o cliente mais do que negociação de contratos e resposta rápida às mudanças mais do que seguir planos \cite{manifesto}. 

A ideia principal é que no desenvolvimento de \textit{software} existem diversas variáveis, quer sejam de natureza ambiental ou técnica, que provavelmente mudarão ao longo da execução do processo. Essa característa torna o processo de desenvolvimento pouco previsível e complexo, requerendo flexibilidade e personalização para ser capaz de responder às mudanças. Outro sim, o contato do cliente deve ser ativo, haja visto que, ele é determinante para a criação do \textit{software}, e por consequência, afeta o próprio processo do serviço, no caso as atividades do desenvolvimento.

A terceirização de serviços em organizações públicas no contexto de contratação de empresas de desenvolvimento de \textit{software} é crescente. A gestão do processo de desenvolvimento de \textit{software} é um grande desafio para essas organizações, pois a maioria delas não são responsáveis diretamente pelo desenvolvimento do \textit{software} e ao mesmo tempo elas precisam, como contratantes, gerenciar o andamento do processo de desenvolvimento do \textit{software} de suas contratadas. 

Com isso, com objetivo de trabalhar a problemática atual de adoção de métodos ágeis na gestão de contratos em um trabalho de conclusão de curso foi realizado um estudo de caso em uma organização pública brasileira, a qual utilizasse métodos ágeis e o  pensamento lean na sua solução de gestão de contrato de fornecedores de desenvolvimento de software, para análise da influência e efeitos do uso dessas metodologias nos resultados da gestão do contrato.
