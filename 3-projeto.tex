\section{Projeto do Estudo de Caso}
\label{sec:projeto}


\subsection{Definição}

Este trabalho consistiu na investigação, coleta, análise e discussão dos resultados de dados da gestão de contrato de um fornecedor de desenvolvimento
 de software para uma organização pública brasileira. O foco foi a análise dos efeitos sobre a entrega de ordens de serviço, a satisfação do cliente e a qualidade interna do código
 do contrato selecionado, o qual é alinhado com os métodos ágeis, com
 o pensamento lean e com a fase de Gerenciamento do Contrato da Instrução Normativa
 nº 04.

Para tanto, foi estruturado neste trabalho problema, questões de pesquisa e objetivos, os quais guiaram o pesquisador durante a fase de coleta de dados.


O problema refere-se ao problema de pesquisa identificado. A questão de pesquisa refere-se a questão de pesquisa que foi derivada do problema. Para responder a questão de pesquisa foram construídos dois GQM's (Goal, Question, Metric) \cite{gqm}. Cada GQM possui um objetivo e questões específicas para coleta de métricas a partir de determinada fonte, a fim de atingir o objetivo e responder a questão de pesquisa do trabalho. Ao todo foram definidas doze questões de pesquisa específicas que serão analisadas no estudo de caso. A definição dessa estrutura está a seguir.


\textbf{Problema:} Alguns contratos de desenvolvimento de \textit{software} da organização não resultaram na entrega do \textit{software} requisitado ao final do contrato.

\textbf{Questão de Pesquisa:} Como o uso de métodos ágeis e do pensamento \textit{lean} na gestão de contratos de fornecedores de desenvolvimento de \textit{software} influenciaram no resultado final do contrato do ponto de vista do gestor de contrato e do fiscal técnico do contrato, que juntos gerenciam o contrato?


\textbf{OE1. Objetivo do Processo:} Analisar a influência do uso de métodos ágeis e do pensamento \textit{lean} na gestão de contrato do contrato do Sistema Integrado de Conhecimento e Gestão (SICG) com a empresa EGL - Engenharia do ponto de vista do gestor de contrato.

\textbf{Questões Específicas do Processo:}

\begin{table}[H]
\footnotesize
\center
\begin{tabular}{|p{0.8cm}|p{8.0cm}|p{7.0cm}|}
\hline
\textbf{ID} & \textbf{Questão}                                                                                                                                  & \textbf{Métrica}    \\ \hline
QE1.        & Qual a quantidade total de ordens de serviço?                                                                                                    &  quantidade total de ordens de serviço         \\ \hline
QE2.        & Qual a quantidade de ordens de serviço que tiveram entrega de \textit{software} funcional?                                                                & quantidade de ordens de serviço de \textit{software}          \\ \hline
QE3.        & Qual a quantidade de ordens de serviço que tiveram entrega apenas de documentação?                                                               & quantidade de ordens de serviço de documentação            \\ \hline
QE4.        & Qual a proporção de entrega de \textit{software} funcional?                                                                                               &  quantidade de ordens de serviço de \textit{software} que tiveram entrega de \textit{software} funcional /quantidade total de ordens de serviço                          \\ \hline
QE5.        & Qual a duração média de entrega de \textit{software} funcional?                                                                                           &  duração total das sprints/duração total das sprints que tiveram entrega de \textit{software} funcional                   \\ \hline
QE6.        & Qual a quantidade de ordens de serviço que não teve entrega de \textit{software} funcional e  de documentação?   &  quantidade de ordens de serviço sem entrega de \textit{software} e de documentação        \\ \hline
QE7.        & Qual a porcentagem de requisitos atendidos em cada ordem de serviço?                                                                             &  (requisitos atendidos/requisitos pedidos) * 100 \\ \hline
QE8.        & Quantas multas foram aplicadas no contrato?                                                                                                      &  quantidade de multas                      \\ \hline
QE9.        & Qual o custo de cada \textit{sprint} e ordem de serviço do projeto?                                                                                       &  custo/sprint               \\ \hline
QE10.        & O quanto de visibilidade do que estava sendo feito o gestor do contrato teve durante o contrato?                                      &  alto, médio ou baixo              \\ \hline
QE11.        & Qual o nível de satisfação com o \textit{software} entregue ao final do contrato?                         &   muito satisfeito, satisfeito, neutro, insatisfeito, muito insatisfeito             \\ \hline
\end{tabular}
\caption{Questões Específicas do Processo}
		\label{qespec}
\end{table}


\textbf{OE2. Objetivo do Produto:}Analisar a qualidade do código fonte com o uso de métodos ágeis e do pensamento \textit{lean} na gestão do contrato do contrato do Sistema Integrado 
de Conhecimento e Gestão (SICG) com a empresa EGL - Engenharia do ponto de vista do  e do fiscal técnico do contrato.

\textbf{Questão Específicas do Produto:}

\begin{table}[H]
\footnotesize
\center
\begin{tabular}{|p{0.8cm}|p{8.0cm}|p{7.0cm}|}
\hline
\textbf{ID} & \textbf{Questão}                                                                                                                                  & \textbf{Métrica}    \\ \hline
QE12.        &  Qual a qualidade interna do produto entregue até o momento ?                                                                            &  bom, excelente, regular e ruim       \\ \hline
\end{tabular}
\caption{Questões Específicas do Produto}
		\label{qespecprod}
\end{table}



\subsection[Background]{Background}

À luz do levantamento bibliográfico realizado, não foram encontrados estudos que buscassem levantar e analisar os efeitos advindos do uso de metodologias ágeis no contexto da gestão de contratos de fornecedores de desenvolvimento de \textit{software} para organizações públicas brasileiras. O Ácordão nº 2314/2013 traz apenas um levantamento do que foi feito de uso de métodos ágeis pelos órgãos analisados. No entanto, não houve uma coleta e análise de dados, tal qual realizada neste trabalho.

\subsection[Design]{Design}

Este estudo de caso é classificado como exploratório, pois não esperamos obter uma resposta definitiva para o problema proposto. A ideia é obter uma visão mais acurada do problema para posteriormente realizar uma pesquisa mais aprofundada. Tal escolha foi feita porque o tema escolhido foi pouco explorado até o momento, constituindo apenas a primeira etapa de uma investigação mais ampla e sistemática. 

Considerando as restrições de tempo para realização deste estudo de caso, foi definido o estudo de um caso único (uma organização) e holístico, com uma unidade de análise (um contrato). 

\subsection[Seleção]{Seleção}

A organização selecionada para este estudo de caso foi o Instituto do Patrimônio Histórico e Cultural (IPHAN). Dentre as organizações analisadas no relatório do Ácordão nº 2314/2013, o IPHAN foi a primeira a adotar o uso de métodos ágeis e pensamento \textit{lean} na gestão de seus contratos. 

\section[Fonte e Método Coleta de Dados]{Fonte e Método de Coleta de Dados}

Os dados foram coletados por meio de entrevistas informais, observações, questionários, análise de documentos dos processos da organização e do repositório de código fonte do contrato do Sistema Integrado de Conhecimento e Gestão (SICG) com a empresa EGL - Engenharia.O Sistema Integrado de Gestao do Conhecimento (SICG) teve como objetivo automatizar o processo de trabalho decorrente da metodologia de inventário, cadastro, normatizaçao, fiscalização, planejamento e análise e gestão do patrimônio material. Esta solução de software foi constrída na linguagem Java com a utilização de frameworks conhecidos no mercado, como por exemplo o VRaptor e Hibernate.

Os questionários tinham o objetivo de coletar dados qualitativos e quantitativos a respeito da organização contratante, do gestor de negócio (cliente) e da empresa contratada no que diz respeito a estrutura organizacional, experiência prévia, satisfação, opiniões, percepções e etc. O dados de observação e entrevistas complementaram os questionários sob o ponto de vista qualitativo.

Os dados quantitativos sobre a execução do processo (solução) foram coletados das 25 ordens de serviço documentadas do projeto. Já os dados da qualidade do código fonte foram coletados de 19 sprints do projeto.

\textbf{Documentos}
\begin{itemize}
\item Processo nº 01450.011592/2010-30, cujo assunto é Contratação de Serviços de Desenvolvimento de \textit{Software} para o Sistema Integrado de Conhecimento e Gestão (SICG). 
\item Processo nº 01450.000845/2012-10, cujo assunto é Gestão de Contrato IPHAN Nº 28/2011 - Desenvolvimento de \textit{Software} para o Sistema Integrado de Conhecimento e Gestão (SICG). 
\item Código Fonte do Sistema Integrado de Conhecimento e Gestão (SICG).
\item Backlog do Produto do Sistema Integrado de Conhecimento e Gestão (SICG).
\end{itemize}

\textbf{Questionários}
\begin{itemize}
\item Objetivo: coletar informações dos envolvidos no projeto por parte do IPHAN,  fiscal técnico do contrato, coordenador do projeto e do gestor do contrato, que também representa o papel de dono do produto, e dos envolvidos no projeto por parte da empresa contratada, a EGL - Engenharia, scrum master e desenvolvedores. 
\item Amostra: foi aplicado o questionário para todos da área de sistemas do IPHAN (3 pessoas) e para 3 integrantes da equipe da EGL-Engenharia.
\end{itemize}

\subsection[Validade]{Validade}

As principais ameaças aplicáveis aos estudos de caso 
são: validade do constructo, validade interna, validade externa e confiabilidade ~\cite{yin}.

A validade de constructo diz respeito ao uso de múltiplas fontes de evidência e seleção de informantes chave. Como apresentado anteriormente, neste trabalho foram utilizadas múltiplas fontes de evidências e os informantes que responderam ao questionário e às entrevistas foram os principais envolvidos no projeto, tanto do órgão contratante quanto da empresa contratada. Além disso, a utilização da técnica GQM direcionou a definição de métricas, procurando-se então a reforçar a validade do constructo.

A validade interna pode ser atingida por meio de diversas estratégias, dentre elas a construção da explanação. Se as conclusões apresentadas pelo estudo de caso correspondem autenticamente a alguma realidade reconhecida pelos próprios envolvidos no projeto, há uma validade interna. O uso de várias fontes de dados e métodos de coleta permite a triangulação, uma técnica para confirmar se os resultados
de diversas fontes e de diversos métodos convergem Para ~\cite{yin}. Dessa forma é possível aumentar a
validade interna do estudo e aumentar a força das conclusões. Nesta pesquisa houve
triangulação de dados  e de metodologia. A triangulação de dados se deu
pelo uso de base de documentos e código organizacionais, questionários e entrevistas para coletar dados. A triangulação
de métodos ocorreu pelo uso de métodos de coleta quantitativos e qualitativos.

A validade externa diz respeito a estabelecer o quanto as descobertas podem ser generalizadas. Pode-se testar a coerência entre os resultados
do estudo e resultados de outras investigações similares. Para isso, ~\cite{yin} recomenda a replicaçao do estudo em múltiplos casos. Como o escopo deste trabalho é o estudo de um único caso, a validade externa não poderá ser atingida.

E, por fim, a confiabilidade diz respeito a mostrar que se o estudo for repetido utilizando as mesmas fontes de dados, poderá obter-se os mesmos resultados. Para isso, a definição deste protocolo do estudo de caso e o 
desenvolvimento de um banco de dados do estudo foram táticas utilizadas para garantir a confiabilidade.


\subsection[Trabalho de Campo]{Trabalho de Campo}

O trabalho de campo é parte fundamental do estudo de caso. Por meio dele é que os dados para análise serão coletados. Neste trabalho, as entrevistas informais e a triagem dos processos foram realizadas em campo. Os processos disponibilizados pelo orgão eram extensos, mais de 4000 páginas. Por isso, a triagem sobre esses processos foi realizada para solicitação de cópias das partes que mais interessavam para o contexto e objetivos de pesquisa deste estudo.

\subsection[Análise]{Análise} 

O processo de análise de dados foi composto por três etapas:
\begin{itemize}
\item Redução: refere-se ao processo de seleção, categorização, simplificação e transformação dos dados obtidos, reduzindo-os ao essencial para proporcionar sua análise e interpretação. Os dados qualitativos foram categorizados em: organização, empresa contratada, projeto e solução. Os dados quantitativos foram categorizados de acordo com o tipo de questão específica, e as categorias são: entrega de ordens de serviço, satisfação do cliente e qualidade interna do código fonte.
\item Exibição: consiste na organização dos dados selecionados de forma a possibilitar a análise sistemática das semelhanças e diferenças e seu inter-relacionamento. Os dados foram exibidos, principalmente, por meio de histogramas, tabelas e texto.
\item Conclusão e verificação: consiste em identificar o significado dos dados, suas regularidades, padrões e explicações. A medida que os resultados foram exibidos, uma discurssão e interpretação inicial foi realizada. Posteriormente, na conclusão final foram agrupados os resultados mais relevantes.
\end{itemize}
